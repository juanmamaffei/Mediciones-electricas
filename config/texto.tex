% O comando abaixo pode ser necessário em documentos que usam versões antigas do pacote "fontspec". Se for começar um novo documento, deixe esse linha comentada e use caracteres Unicode.
% \usepackage[utf8]{luainputenc}


% Licencia CC
\usepackage[
    type={CC},
    modifier={by-sa},
    version={3.0},
    lang={sp},
]{doclicense}
% Tamanho do papel e das margens: 

\usepackage[paperheight=230mm,paperwidth=160mm,inner=20mm,outer=14mm,top=24.5mm,bottom=21mm,pdftex]{geometry}

% Suporte a idiomas:

\usepackage[greek,brazilian,spanish]{babel}

% Para usar as variáveis \theauthor, \thetitle e \thedate (autor, título e data).

\usepackage{titling}

% Título de capítulo elegante:

\usepackage[Lenny]{fncychap}

% Pacote para estilo elegante:

\usepackage{fancyhdr}

% Definição de novas cores:

\usepackage{xcolor}

\definecolor{verde}{rgb}{0.2, 0.50, 0.25}
\definecolor{verde_UnB}{cmyk}{1,0,1,0.4}
\definecolor{cinza_UnB}{rgb}{0.6,0.6,0.6} % https://www.ginifab.com/feeds/pms/color_picker_from_image.php


% Para mudar cor de títulos (https://tex.stackexchange.com/a/75670/91816):

\usepackage{sectsty}

\chapterfont{\color{verde_UnB}}  % sets colour of chapters
\sectionfont{\color{verde_UnB}}  % sets colour of sections

% Pacote para definir fonte de qualquer tamanho:

\usepackage{fix-cm}

%% Pacote para correta separação silábica de palavras com hífen;
%% hífens devem ser escritos como \Hyphdash ou \-/ (se opção shortcuts estiver ativa)
%% no texto; por exemplo cana\-/de\-/açúcar seria separada como
%% cana-de- numa linha e -açúcar na linha seguinte
%% repetindo o hífen (um hífen é da separação e outro é da palavra):

\usepackage[shortcuts]{extdash}

%% Definindo espacamento:

\usepackage{setspace} 
\singlespacing

%% Enumerar itens:

\usepackage{enumitem}

%% Pacote para criar caixas:

\usepackage{pbox} %  que limitam a largura de textos em células,
%\usepackage{minibox} % que têm largura arbitrária.

%% Definição das dimensões do texto:

%\setlength{\textwidth}{16cm}
%\setlength{\textheight}{22cm}
%\setlength{\headheight}{1cm}
%\setlength{\footheight}{1cm}

\usepackage[title,titletoc]{appendix}

%% Passando títulos para o português:

\renewcommand{\chaptername}{Capítulo}
\renewcommand{\bibname}{Referencias}
\renewcommand{\appendixname}{Anexo}
\renewcommand{\indexname}{Índice}
\renewcommand{\contentsname}{\bf\color{verde_UnB}Contenidos}
\renewcommand{\tablename}{Tabla}
\renewcommand{\figurename}{\bf Fig.}
\renewcommand{\sin}{sen}
\def\listoffiguresname{Lista de Figuras}
\def\listoftablesname{Lista de Tablas}

%% Definindo estilo de pagina:

\pagestyle{fancy}

% PACOTES DE TABELAS:

%% Espaçamento de tabelas ajustado:

\usepackage{booktabs}

%% Pacotes de linhas e colunas multiplas:

\usepackage{multirow}
\usepackage{multicol}

%% Pacotes de tabela e tabulação:

%%% Tabela com caixas centralizada:

\usepackage{array}
\newcolumntype{P}[1]{>{\centering\arraybackslash}p{#1}}

%%% Tabela em múltiplas páginas:

\usepackage{longtable}

%%% Tabela colorida:

\usepackage{tabu}
\usepackage{colortbl}

%%% Mudando cor das linhas de todas as tabelas:

\makeatletter
\renewenvironment{table}
     {\@float{table}\taburulecolor{verde_UnB}\arrayrulecolor{verde_UnB}}
     {\end@float}
\makeatother

%%% Tabela com linhas pontilhadas:

\usepackage{arydshln}

% PACOTES MATEMÁTICOS:

%% Pacote para não-itálico em ambiente de fórmulas:
%\usepackage[]{mathastext}

%% Pacotes matemáticos:

\usepackage{amsmath}
\usepackage{mathtools}

%%% Numera apenas equações usadas:

%\mathtoolsset{showonlyrefs=true}

%%% Declarando barras para valor absoluto e norma:

\DeclarePairedDelimiter\abs{\lvert}{\rvert}%
\DeclarePairedDelimiter\norm{\lVert}{\rVert}%

%% Nome de equações ao lado com comando "eqname":

\newcommand{\eqname}[1]{\tag*{\llap{#1}}}

%% Fontes tipográficas:

\usepackage{fontspec}
%\usepackage{libertineotf}

%%% Seleção da fonte UnB Pro:

\setmainfont{UnB-Pro}[
  Path = fonts/UnB_Pro_v1.0/,
  Extension=.otf,
  UprightFont=*_Regular,
  ItalicFont=*_Italic,
  BoldFont=*_Bold,
  BoldItalicFont=*_Bold-Italic,
]

\setmonofont{Libertinus Mono}

%%% Para incluir o comando \url{}:

\usepackage[breaklinks=true, hidelinks]{hyperref}

%% Configurações ABNTeX:

\usepackage[hyphens,alf,abnt-and-type=e,abnt-last-names=bibtex,abnt-etal-cite=1,abnt-etal-list=1]{abntex2cite}


% Recuo no primeiro parágrafo:

\usepackage{indentfirst}

% Mudando recuo de parágrafo e tamanho da fonte:

\setlength{\parindent}{7mm}
\fontsize{11pt}{13.2pt}\selectfont

% Usado para evitar linhas órfãs:

\usepackage{needspace}

% Mudar espaçamento entre número e título de seção:

%% \def\l@figure{\@dottedtocline{1}{1.5em}{2.5em}}

\usepackage{tocloft}
\setlength{\cftfignumwidth}{2.55em} 

% Para escrever "Apêndices" e "Anexos":

%\usepackage[titletoc]{appendix}

% Elimina recuo nas notas de rodapé:

\usepackage[hang,flushmargin]{footmisc}

% Primeira página vazia em cada capítulo:

\makeatletter
\renewcommand\chapter{\if@openright\cleardoublepage\else\clearpage\fi
                    \thispagestyle{empty}% original style: plain
                    \global\@topnum\z@
                    \@afterindentfalse
                    \secdef\@chapter\@schapter}
\makeatother

% Diminui chance de linhas órfãs e viúvas:

\clubpenalty1000000
\widowpenalty1000000

% Comando para mudar a fonte de citação literal de código (verbatim):

\makeatletter
\newcommand{\verbatimfont}[1]{\def\verbatim@font{#1}}%
\makeatother

% Pacote para referir a capítulos pelo nome:
\usepackage{nameref}
\makeatletter
\newcommand*{\currentname}{\@currentlabelname}
\makeatother

% Definições (alternativas a nameref) para se referir a capítulos e seções pelo nome:
\let\Chaptermark\chaptermark
\def\chaptermark#1{\def\Chaptername{#1}\Chaptermark{#1}}
\let\Sectionmark\sectionmark
\def\sectionmark#1{\def\Sectionname{#1}\Sectionmark{#1}}
\let\Subsectionmark\subsectionmark
\def\subsectionmark#1{\def\Subsectionname{#1}\Subsectionmark{#1}}
\let\Subsubsectionmark\subsubsectionmark
\def\subsubsectionmark#1{\def\Subsubsectionname{#1}\Subsubsectionmark{#1}}

% Pacote para usar "clearpage" sem finalizar página, apenas descarregar todos os elementos já inseridos:

\usepackage{afterpage}

% Pacote para que páginas em branco não mostrem cabeçalhos nem rodapés:

\usepackage{emptypage}

% Pacote para índice Remissivo (a ser impresso por \printindex):

\usepackage{makeidx}
\makeindex
