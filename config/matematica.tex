% Tamanhos de fontes matemáticas em relação à fonte do texto:
%% {tamanho do texto} {matemática} {matemática script} {matemática scriptscript}

\DeclareMathSizes{10}{10}{6}{4}
\DeclareMathSizes{9}{9}{5}{3}

\usepackage{amsthm} % deve ser chamado antes de mdframed conforme dito em http://tex.stackexchange.com/questions/283763/why-dont-i-get-non-italic-normal-font-inside-theorem-environment-using-newmdth

% Seleção de fontes de equações:

\usepackage[math-style=french]{unicode-math} % não funciona se compilar com pdflatex, deve-se usar xelatex ou lualatex

\usepackage{yfonts} % para usar caracteres góticos para algumas variaveis

% Pacote para maior controle de fluxo de figuras (usado para opção [H] que fixa a posição das figuras):

\usepackage{float}

% Pacote usado para impedir elementos flutuantes (figuras, tabelas, etc.) de aparecerem em seção errada:

\usepackage[section]{placeins}

% Pacote inserido para fazer o símbolo de grau:

\usepackage{gensymb}

% Pacote inserido para fazer símbolos de circuito elétrico:

\usepackage{marvosym}

% Definição de variáveis e unidades:

\newcommand{\angstrom}{\text{\normalfont\AA}}
\newcommand{\pol}{\ensuremath{pol}}
\newcommand{\cm}{\ensuremath{cm}}
\newcommand{\km}{\ensuremath{km}}
\newcommand{\hm}{\ensuremath{hm}}
\newcommand{\dam}{\ensuremath{dam}}
\newcommand{\dm}{\ensuremath{dm}}
\newcommand{\mol}{\ensuremath{mol}}
\newcommand{\mi}{\ensuremath{mi}}
\newcommand{\h}{\ensuremath{h}}
\newcommand{\s}{\ensuremath{s}}
\newcommand{\Min}{\ensuremath{min}}
\newcommand{\Pa}{\ensuremath{Pa}}
\newcommand{\atm}{\ensuremath{atm}}
\newcommand{\mmHg}{\ensuremath{mmHg}}
\newcommand{\BarP}{\ensuremath{Bar}}
\newcommand{\PsiP}{\ensuremath{PSI}}
\newcommand{\lb}{\ensuremath{lb}}
\newcommand{\N}{\ensuremath{N}}
\newcommand{\kg}{\ensuremath{kg}}
\newcommand{\kgf}{\ensuremath{kgf}}
\newcommand{\are}{\ensuremath{a}}
\newcommand{\litro}{\ensuremath{\ell}}
\newcommand{\g}{\ensuremath{g}}

% Símbolos de diferencial:

\def\D{\mathrm{d}} %% diferencial - comando \D{}
\def\DI{{\delta}} % diferencial inexata ``delta''

\newcommand*\diff{\mathop{}\!\mathrm{d}}
\newcommand*\Diff[1]{\mathop{}\!\mathrm{d^#1}}

% Evita que equações extrapolem a margem (e permite outros recursos tipográficos avançados):

\usepackage{microtype}

%% Simbolos matemáticos extra:

\DeclareMathSymbol{\Omega}{\mathalpha}{letters}{"0A}
\DeclareMathSymbol{\varOmega}{\mathalpha}{operators}{"0A}
\providecommand*{\upOmega}{\varOmega} % for siunitx

%% Pacote para uso de unidades SI com distância padronizada entre valor e unidade:

\usepackage{siunitx}
\sisetup{%
      binary-units=true,
      group-separator={.},
      group-digits=integer,
      load-configurations=abbreviations,
      load=addn,
      per-mode=fraction,
      output-decimal-marker={,},
      range-phrase= --,
      separate-uncertainty=true,
      math-ohm = \ensuremath{\upOmega}, % senão \ohm não funciona
      text-ohm = Ω,  % senão \ohm não funciona
    }

\DeclareSIUnit\milha{mi}
\DeclareSIUnit\polegada{pol}
\DeclareSIUnit\alqueire{alqueire}
\DeclareSIUnit\inch{in}
\DeclareSIUnit\foot{ft}
\DeclareSIUnit\kgf{kgf}
\DeclareSIUnit\lbf{lbf}
\DeclareSIUnit\pound{lb}
\DeclareSIUnit\rev{rev}
\DeclareSIUnit\rpm{rpm}
\DeclareSIUnit\pkt{PKT}
\DeclareSIUnit{\calorie}{cal}
\DeclareSIUnit{\cal}{cal}
\DeclareSIUnit{\Cal}{Cal}
\DeclareSIUnit{\Calorie}{\kilo\calorie}
\DeclareSIUnit{\fahrenheit}{\degree F}

\DeclareSIUnit{\nothing}{\relax} % usado para mostrar prefixos sem unidade
