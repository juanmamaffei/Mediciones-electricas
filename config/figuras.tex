% Pacotes auxiliares no design da capa:

% Códigos QR
\usepackage{qrcode}

%Por defecto:\quad
%\qrcode{https://www.jmmaffei.com}
%\qquad

%+1" de alto y de largo:
%\quad
%\qrcode[height=1in]{https://www.ctan.org/tex-archive/macros/latex/contrib/qrcode?lang=en}

\usepackage{xcoffins}
%\usepackage{svg}
%% Pacotes para usar o pacote Metapost para fazer graficos e diagramas:

\usepackage{luamplib}
\everymplib{input mpcolornames; beginfig(1);}
\everyendmplib{endfig;}
% Paquete para imágenes Postscript
%\usepackage{epsf} 
%% Pacotes para diagramas, desenhos, gráficos:


\usepackage{tikz}
\usetikzlibrary{backgrounds,patterns}

\usepackage{pgfplots}
\pgfplotsset{compat=1.12}

%% Pacote para ferramentas gráficas:

\usepackage{graphics}
\usepackage{graphicx} % Sem esse, alguns includegraphics não funcionam.


\usepackage[pdf]{pstricks}
\usepackage{pst-node,pst-circ,pst-plot,pst-3dplot,pst-all}

\usepackage{threeparttable} % Para alinhar figuras e legendas com "measuredfigure".

\usepackage{wrapfig} % Figuras ao lado do texto.

\usepackage{picinpar}% Alteranativa para figuras ao lado de texto. http://ctan.org/pkg/picinpar

\usepackage[font={small}, margin=0cm, justification=centering]{caption} % Legendas com fonte pequena.

% Comando para adicionar fonte de figuras e semelhantes:

\newcommand{\source}[1]{\captionsetup{singlelinecheck=false,justification=justified}\caption*{\footnotesize \noindent Fonte: {#1}}}