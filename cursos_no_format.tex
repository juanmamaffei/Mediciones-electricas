\documentclass[11pt,a4paper]{book}
\usepackage[utf8]{inputenc}
\usepackage[spanish]{babel}
\usepackage{amsmath}
\usepackage{amsfonts}
\usepackage{amssymb}
\usepackage{graphicx}

\usepackage[pdf]{pstricks}
\usepackage{pst-node,pst-circ,pst-plot,pst-3dplot,pst-all}

\author{Prof. Juan Manuel Maffei}
\title{Curso de mediciones eléctricas}
\begin{document}

	\newtheorem{ejemplo}{Ejemplo}[chapter]
	% \tableofcontents
		
	\chapter{Clase 1: Diagnóstico e introducción a las mediciones}
	
	Para presentarme brevemente, mi nombre es Juan Manuel Maffei, soy Técnico en Electrónica y otras cosas. Tanto mi formación académica como mi experiencia son en su mayor parte dedicadas a la docencia. Actualmente trabajo en esta escuela, en materias técnicas de sexto año, de la carrera de Electromecánica, en el Instituto de Profesorado como Jefe de la sección Matemática y como docente de materias del Profesorado de Matemática y en una escuela técnica de Rosario como MET de Electricidad. Fuera de la docencia, hago infraestructura tecnológica para empresas de la zona y desarrollo sistemas informáticos.
		
	En estas clases vamos a realizar una serie de prácticas utilizando instrumentos de medición eléctrica, pero siempre tratando de comprender en todo momento qué es lo que estamos midiendo, por qué lo estamos midiendo, para qué nos puede servir esa medición, qué precisión tiene ese experimento que estamos realizando y cómo este tipo de experiencias puede servirnos para detectar fallas o verificar el correcto funcionamiento de una instalación eléctrica.
	
	Yo no me las sé todas, simplemente soy un técnico que tiene bastante práctica en estos temas que vamos a tratar y por lo tanto puedo meter la pata, puede que alguno de ustedes haya trabajado en algo parecido y pueda sumarnos su experiencia, y no hay ningún problema con eso. De hecho, estaría buenísimo que todos podamos enseñarnos entre todos. Más que un docente, voy a tratar de ser una especie de coordinador o de guía para que podamos aprender juntos.
	
	Esta es una organización a priori de las clases. La idea es desarrollar estos bloques de temas, lo cual se puede ir modificando a medida que vayan pasando las horas. Qué les parece? Les suena? No tienen idea? Es muy fácil?
	
	Bueno, justamente para que me indiquen qué les parece y para saber qué es lo que están buscando en este curso, les voy a dejar este diagnóstico para que hagamos en 10 minutos, así puedo ajustar la organización del curso a sus necesidades.
	
	La semana que viene ya les voy a dejar el apunte final, porque estoy retocando algunas cosas y me demoré con ese tema. Por el momento les voy a dejar un sitio web donde se pueden inscribir y seguir todas las clases.
	
\end{document}