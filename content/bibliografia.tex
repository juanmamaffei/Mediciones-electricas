\begin{thebibliography}{X}
	% \bibitem{Ref2} \textsc{Autores}, (AÑO). \textit{Título}, ciudad, editorial.	
	\bibitem{boylestad} \textsc{Boylestad, R.}, (2011). \textit{Introducción al análisis de circuitos} Decimosegunda edición, México, Pearson Educación.
	\bibitem{hewitt} \textsc{Hewitt, P.} (2002). \textit{Conceptual physics}. Pearson Educación.
	\bibitem{bolton} \textsc{Bolton, W.} (1995). \textit{Mediciones y pruebas eléctricas y electrónicas}. Marcombo.
	\bibitem{uniandes} \textsc{Calderón, J.} (2006). \textit{Fundamentos de las Mediciones Eléctricas. Teorı́a y Prácticas de Laboratorio}. Escuela de Ingenierı́a Eléctrica, Universidad de Los Andes.
	\bibitem{utnmendoza} \textsc{Samsó, F.} (2008). \textit{Apuntes de Cátedra de Máquinas e Instalaciones Eléctricas}. Departamento de Electŕonica. Universidad Tecnológica Nacional: Facultad Regional Mendoza.
	\bibitem{maqelec} \textsc{Ortega, G.; Gómez, M. \& Bachiller, A.} (2002). \textit{Problemas Resueltos de Máquinas Eléctricas}. Thomson. Madrid.
	\bibitem{suarez} \textsc{Suárez, J. A.} (2006). \textit{Medidas Eléctricas: segunda edición}. Libro de cátedra de Mediciones Eléctricas I: Facultad de Ingeniería, Universidad Nacional de Mar del Plata.
	\bibitem{frank} \textsc{Frank, E.} (1969). \textit{Análisis de Medidas Eléctricas}. McGraw Hill. Madrid.
	\bibitem{aeabt} \textsc{Asociación Electrotécnica Argentina} (2007). \textit{Documento Normativo 95150. Suministro y medición en baja tensión}. AEA.
	\bibitem{aeamantenimiento} \textsc{Asociación Electrotécnica Argentina} (2006). \textit{Documento Normativo 90706 Guía para la gestión del Mantenimiento en instalaciones. }. AEA.
	\bibitem{aeamantenimiento} \textsc{Asociación Electrotécnica Argentina} (2006). \textit{Documento Normativo 90364-7-771 Reglamentación para la ejecución de instalaciones eléctricas en inmuebles – Viviendas, oficinas y locales (unitarios). }. AEA.
	%Materiales del curso de EDx
	
\end{thebibliography}