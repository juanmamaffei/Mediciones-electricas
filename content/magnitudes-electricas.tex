\chapter{Magnitudes eléctricas}

En este capítulo, arribaremos a definir \textbf{intensidad de corriente eléctrica}, \textbf{tensión eléctrica}, \textbf{resistencia eléctrica}, y veremos cómo estas tres magnitudes se encuentran relacionadas. Para ello, será necesario contar con nociones acerca de los átomos y su estructura. Tranquilo... no nos detendremos demasiado en estas ideas, pero debés saber que son fundamentales para poder construir una base sólida en la comprensión de los fenómenos eléctricos.

\section{El átomo}

La materia está constituida por unidades muy pequeñas (del orden de los picómetros) llamadas \textbf{átomos}.

Para tener una noción más clara de la pequeñez de los átomos, \textit{el diámetro de un átomo es al diámetro de una manzana como el diámetro de una manzana es al diámetro de la Tierra}.

A su vez, estos átomos poseen un \textbf{núcleo}, formado por \textbf{protones} y \textbf{neutrones}, y uno o varios \textbf{electrones}; todos atraídos o repelidos por lo que se conoce con el nombre de \textbf{fuerza eléctrica}.

Imaginemos por un segundo que en el Universo, todas las partículas se atrayeran entre sí... Toda la materia estaría comprimida en un único bloque compacto.

Si, por el contrario, todas las partículas se repelieran, el Universo sería un gas en permanente expansión.

Es deducible entonces, que estos tres tipos de partículas (protones, neutrones y electrones), se encuentran atraídas o repelidas entre sí de manera equilibrada para que el Universo pueda subsistir tal y como lo conocemos.

Se dice que los protones tienen carga eléctrica positiva, que repele cargas positivas pero que atrae cargas negativas. Entonces, estos protones positivos en el núcleo, atraen a una nube de electrones negativos a su alrededor para constituir el átomo.

Los electrones, con carga negativa, son atraídos por el núcleo positivo de protones, pero se rechazan entre sí. Son muy ligeros y se mueven muy rápido. 

Debido a este rechazo entre los electrones, no atravesamos una pared al tocarla.

Si bien la masa de un protón es mucho mayor a la de un electrón, ambos poseen la misma carga eléctrica (es decir que un electrón es tan negativo como positivo es un protón). Por este motivo, las fuerzas eléctricas de atracción y repulsión, harán que las cargas de un átomo se encuentren balanceadas. En otras palabras: un átomo tendrá la misma cantidad de protones que de electrones, y entonces es eléctricamente neutro.

Aunque los átomos tengan carga neutra, a veces los electrones de un átomo son atraídos por el núcleo de otros átomos, provocando una "asociación" que da lugar a la formación de moléculas.

\subsection{Fuerzas eléctricas}

Según la cantidad de electrones que se encuentren atraídos al núcleo de un átomo, se organizarán en diferentes capas o niveles de energía. Mientras la distancia al núcleo sea mayor, la \textbf{fuerza eléctrica} será menor.

La fuerza eléctrica es una magnitud vectorial, que aparece entre dos cuerpos cargados eléctricamente, y puede ser de atracción o de repulsión.

Para comprender mejor la fuerza eléctrica, podemos recurrir a una analogía con la fuerza magnética, con la que hemos experimentado toda la vida: 
A medida que acercamos un imán fijo a un metal, podemos percibir que la fuerza de atracción entre ambas piezas es mayor, y a medida que lo alejamos, la fuerza es menor.

La fuerza eléctrica entre dos cuerpos cargados se calcula mediante la Ley de Coulomb:

$$ F = k \times \frac{q_1 \times q_2}{d^{2}} $$

\begin{itemize}
	\item $k=9 \times 10^{9}\frac{N.m^{2}}{C^{2}}$
	\item $q_1$ y $q_2$ son las cantidades de carga de ambos cuerpos, expresados en Coulombs.
	\item $d$ es la distancia que separa ambos cuerpos.
\end{itemize}

Con respecto a la unidad de carga, \textit{1 Coulomb} equivale a aproximadamente $6,241509\times 10^{18}$ electrones, o $6,24$ millones de billones de electrones... que si bien es una cantidad enorme, sólo representa a la carga que pasa por un cargador de teléfono celular durante unos pocos segundos.

\section{Materiales conductores y aislantes}

Aplicando esta Ley, puede deducirse que los electrones de un átomo pueden estar fuerte o débilmente atraídos por el núcleo según la distancia que mantengan con respecto al núcleo.

Si la atracción entre el núcleo y alguno de sus electrones es débil, ese material es conductor. Si, por el contrario, todos los electrones del átomo se encuentran fuertemente atraídos por el núcleo, no se podrán desprender con facilidad del átomo y por lo tanto son materiales aislantes.

Pero la atracción de los electrones con los núcleos de sus respectivos átomos no depende sólo de su "cercanía al núcleo" sino también de la completud de las capas: los átomos con capas completas suelen ser estables y los átomos con capas incompletas suelen tener más facilidad para ceder electrones (recordando que un material que cede electrones con facilidad es un conductor).

Una capa de electrones puede contener hasta $2n^{2}$ electrones, donde $n$ es el número de capa.

Tomemos como ejemplo el cobre, cuyos átomos poseen 29 protones y 29 electrones.

Los electrones se encuentran distribuidos de la siguiente forma:
\begin{itemize}
	\item Capa 1: 2
	\item Capa 2: 8
	\item Capa 3: 18
	\item Capa 4: 1
\end{itemize}

El electrón de la cuarta capa, se encuentra alejado del núcleo, y además, está en una capa incompleta, porque la cuarta capa puede contener hasta $2.4^{2}=32$ electrones, y por estos motivos el cobre es buen conductor de corriente eléctrica.

Si hubiera, en las inmediaciones del átomo de cobre, una fuerza de atracción lo suficientemente fuerte, el electrón de la cuarta capa se liberaría del átomo padre, quedando el átomo con 29 protones y 28 electrones y carga positiva. Cuando esto ocurre, el átomo se convierte en un ion positivo.

Algo similar ocurre en todas las baterías: una separación de cargas positivas y cargas negativas, a través de medios químicos.

Una fuente de tensión no es más que un dispositivo con regiones de carga positiva y carga negativa. Mientras mayores sean estas cargas, mayor será la tensión o voltaje.

Para efectuar este proceso de separación de cargas, es necesario gastar cierta energía. Así, el potencial eléctrico se define como el trabajo necesario para mover estas cargas.

$$ \text{Potencial eléctrico} = \frac{\text{energía potencial}}{\text{carga}} $$

Si utilizamos las unidades del S.I., se dice que \textit{1 Volt de potencial equivale a 1 Joule de de energía por 1 Coulomb de carga}

$$ 1 V = 1 \frac{J}{C} $$