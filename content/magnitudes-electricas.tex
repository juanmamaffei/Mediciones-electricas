\chapter{Magnitudes eléctricas}

En este capítulo, arribaremos a definir \textbf{intensidad de corriente eléctrica}, \textbf{tensión eléctrica}, \textbf{resistencia eléctrica}, y veremos cómo estas tres magnitudes se encuentran relacionadas. Para ello, será necesario contar con nociones acerca de los átomos y su estructura. Tranquilo... no nos detendremos demasiado en estas ideas, pero debés saber que son fundamentales para poder construir una base sólida en la comprensión de los fenómenos eléctricos.

\section{El átomo}

La materia está constituida por unidades muy pequeñas (del orden de los picómetros) llamadas \textbf{átomos}.

Para tener una noción más clara de la pequeñez de los átomos, \textit{el diámetro de un átomo es al diámetro de una manzana como el diámetro de una manzana es al diámetro de la Tierra}.

A su vez, estos átomos poseen un \textbf{núcleo}, formado por \textbf{protones} y \textbf{neutrones}, y uno o varios \textbf{electrones}; todos atraídos o repelidos por lo que se conoce con el nombre de \textbf{fuerza eléctrica}.

Imaginemos por un segundo que en el Universo, todas las partículas se atrayeran entre sí... Toda la materia estaría comprimida en un único bloque compacto.

Si, por el contrario, todas las partículas se repelieran, el Universo sería un gas en permanente expansión.

Es deducible entonces, que estos tres tipos de partículas (protones, neutrones y electrones), se encuentran atraídas o repelidas entre sí de manera equilibrada para que el Universo pueda subsistir tal y como lo conocemos.

Se dice que los protones tienen carga eléctrica positiva, que repele cargas positivas pero que atrae cargas negativas. Entonces, estos protones positivos en el núcleo, atraen a una nube de electrones negativos a su alrededor para constituir el átomo.

Los electrones, con carga negativa, son atraídos por el núcleo positivo de protones, pero se rechazan entre sí. Son muy ligeros y se mueven muy rápido. 

Debido a este rechazo entre los electrones, no atravesamos una pared al tocarla.

Si bien la masa de un protón es mucho mayor a la de un electrón, ambos poseen la misma carga eléctrica (es decir que un electrón es tan negativo como positivo es un protón). Por este motivo, las fuerzas eléctricas de atracción y repulsión, harán que las cargas de un átomo se encuentren balanceadas. En otras palabras: un átomo tendrá la misma cantidad de protones que de electrones, y entonces es eléctricamente neutro.

Aunque los átomos sean neutros, a veces los electrones de un átomo son atraídos por el núcleo de otros átomos, provocando una "asociación" de átomos, que dan lugar a la formación de moléculas.