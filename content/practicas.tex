\chapter{Prácticas de laboratorio}

\section{Mediciones en CC}
\subsection{Prueba de componentes}
Diodos, fusibles, resistencias, inductancias (con ohmetro, con medidor de inductancias), capacitores.

Conexión de puente rectificador a diferentes tensiones.

Mediciones de resistencia interna de fuentes de CC y baterías.

Medición de tensiones de fuentes de alimentación continua.

Medición de corrientes continuas.

Medición de tensiones en un circuito de corriente continua.

Medición de resistencia en resistores.

Ensayos de resistencia equivalente


Medición de inductancia.

Medición de capacidad.

Transitorios de capacitores e inductores

\section{Ensayos sobre instalaciones de CA}
\subsection{Rectificación de CA}
Puente rectificador con diodos LED, alimentado por el generador de funciones y por una fuente de corriente continua.

Rectificación de corriente alterna. Mediciones con multímetro y osciloscopio.

Corrección de rizado. Mediciones con osciloscopio.

\subsection{Consumos}
\subsection{Energía}
\subsection{Puesta a tierra}
\subsection{Potencia}
activa, reactiva, aparente, factor de potencia.
\subsection{Armónicas}
Lámparas de distinto tipo, cargas de distinto tipo, fuente de PC, fluorescente, led, etc..

\section{Ensayos sobre máquinas eléctricas}
\subsection{Motores de CA}
\subsubsection{Aislamiento}
\subsubsection{Bobinados}
\subsubsection{Frecuencia}
\subsection{Transformadores}
	\subsubsection{Transformador de aislación galvánica}
	Usando el gabinete de pruebas...
\section{Tratamiento digital de datos}