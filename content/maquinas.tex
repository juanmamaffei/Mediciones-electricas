\chapter{Máquinas eléctricas}
En este capítulo se desarrolla una explicación general acerca de los tres grandes tipos de máquinas eléctricas.

Se explicarán los principios de funcionamiento y se analizarán algunas ecuaciones que serán de ayuda para realizar ensayos durante este curso. Debe tenerse en cuenta que el estudio de máquinas eléctricas requiere de un curso por sí mismo (como mínimo), por lo que el estudio que se realizará aquí, no es para nada exhaustivo.

\section{Transformadores}

Un transformador es una máquina eléctrica que cuenta con dos devanados concatenados mediante un circuito eléctrico de alta permeabilidad magnética. Existirá un flujo magnético que inducirá una fuerza electromotriz entre sus devanados, relacionados por el número de espiras entre ambas.
\begin{equation}
	\label{eq:relacion_espiras_transformador}
	\frac{N_1}{N_2}=\frac{V_1}{V_2}
\end{equation}
Siendo $N_1$ y $N_2$ el número de espiras de los devanados primario y secundario respectivamente, y $V_1$ y $V_2$ los valores de tensión eficaz obtenidos en cada uno de los devanados.
\section{Generadores}
\section{Motores}
